\chapter{データ構造 - Data structures}

\index{データ構造 -data structure}

\key{データ構造 -data structure}は、コンピュータのメモリにデータを格納するための方法です。
データ構造にはそれぞれ長所と短所があり、あるデータを格納する際に問題に適したデータ構造を選択できるかが大切になります。
この判断に必要な知識は選択したデータ構造でどのような操作が効率的に行えるのかということです。

この章では、C++標準ライブラリの中で最も重要なデータ構造について紹介します。
標準ライブラリを可能な限り使用することは、多くの時間を節約することができるのでベストな方法です。
このドキュメントの後半では標準ライブラリでは利用できないより高度なデータ構造について学びます。

\section{動的配列 - Dynamic arrays}

\index{動的配列 - dynamic array}
\index{配列 - vector}

\key{動的配列 - dynamic array}とは、プログラムの実行中にサイズを変更することができる配列のことです。
C++で最もよく使われる動的配列は\texttt{vector}で,普通の配列と同じように使うことができます.
次のコードは、空のvectorを作成し、そこに3つの要素を追加します。

\begin{lstlisting}
vector<int> v;
v.push_back(3); // [3]
v.push_back(2); // [3,2]
v.push_back(5); // [3,2,5]
\end{lstlisting}

これには配列のようにアクセスが出来ます。
\begin{lstlisting}
cout << v[0] << "\n"; // 3
cout << v[1] << "\n"; // 2
cout << v[2] << "\n"; // 5
\end{lstlisting}

\texttt{size}によってvectorの長さを知ることが出来ます。
これを利用して全ての配列の要素にアクセスすることができます。
\begin{lstlisting}
for (int i = 0; i < v.size(); i++) {
    cout << v[i] << "\n";
}
\end{lstlisting}

\begin{samepage}

また、forを使って次のようにアクセスをすることもできます。
\begin{lstlisting}
for (auto x : v) {
    cout << x << "\n";
}
\end{lstlisting}
\end{samepage}

\texttt{back}によって配列の最後の要素にアクセスすることができ、
\texttt{pop\_back}によって最後の要素を削除することが出来ます。
\begin{lstlisting}
vector<int> v;
v.push_back(5);
v.push_back(2);
cout << v.back() << "\n"; // 2
v.pop_back();
cout << v.back() << "\n"; // 5
\end{lstlisting}

初期値を設定して生成することもできます。
\begin{lstlisting}
vector<int> v = {2,4,2,5,1};
\end{lstlisting}

次のようにして要素数を指定して作成したり、各要素の初期値を決めることが出来ます。
\begin{lstlisting}
// size 10, initial value 0
vector<int> v(10);
\end{lstlisting}
\begin{lstlisting}
// size 10, initial value 5
vector<int> v(10, 5);
\end{lstlisting}

vectorの内部実装は、通常の配列を使用します。
vectorのサイズが大きくなり過ぎる、あるいは小さくなり過ぎるとき、
新しい配列のメモリ確保して全要素を新しい配列に移動させます(訳註:この操作は$O(N)$かかります)。
ただし、このサイズ生成はがあまり起こるものではないので\texttt{push\_back}の平均的な時間複雑度は$O(1)$と考えてよいです。

\index{string}

文字列型である\texttt{string}構造体は動的な配列で、ほとんどvectorのように使用することができます。
stringには他のデータ構造にはない特別な操作ができます。
文字列は \texttt{+} 記号を使って結合することができます。
関数$\texttt{substr}(k,x)$は位置$k$からの長さ$x$の部分文字列を返し,
関数$\texttt{find}(\texttt{t})$はある文字列$t$が最初に出現する位置(訳註:あるいは存在しない)を探します。

次のコードでは、いくつかの文字列操作を紹介します。

\begin{lstlisting}
string a = "hatti";
string b = a+a;
cout << b << "\n"; // hattihatti
b[5] = 'v';
cout << b << "\n"; // hattivatti
string c = b.substr(3,4);
cout << c << "\n"; // tiva
\end{lstlisting}

\section{set - Set structures}

\index{set}

\key{set}は、要素の集合のデータ構造です。
集合の基本操作は、要素の挿入、検索、削除からなります。

C++の標準ライブラリには、2つの集合の実装があります。
1つ目は\texttt{set}で、内部のデータは平衡二分木で格納され,操作は$O(\log n)$時間で行われます。
もう一つは\texttt{unordered\_set}内部のデータをハッシュで格納し、操作は平均して$O(1)$時間で動作する。

どのセット実装を使うかは、好みの問題と特徴を考えて判断します。
\texttt{set}の利点は、要素の順序を維持し、
\texttt{unordered\_set}では利用できない関数を提供されます。
もちろん\texttt{unordered\_set}の方が効率的な場合もある。

次のコードは、整数を含む集合を作成し、その操作の一部を示します。
関数\texttt{insert}は集合に要素を追加し,
関数\texttt{count}は集合内の要素の出現回数を返し,
関数\texttt{erase}は集合から要素を削除する関数です。

\begin{lstlisting}
set<int> s;
s.insert(3);
s.insert(2);
s.insert(5);
cout << s.count(3) << "\n"; // 1
cout << s.count(4) << "\n"; // 0
s.erase(3);
s.insert(4);
cout << s.count(3) << "\n"; // 0
cout << s.count(4) << "\n"; // 1
\end{lstlisting}

setの操作のいくつかはvectorのように使うことができますが、
\texttt{[]}記法を使ってインデックスで指定した要素に直接アクセスすることはできません。
次のコードは、セットを作成し、その中の要素数を表示し、すべての要素に処理をするものです。
\begin{lstlisting}
set<int> s = {2,5,6,8};
cout << s.size() << "\n"; // 4
for (auto x : s) {
    cout << x << "\n";
}
\end{lstlisting}

集合の重要な特性は、そのすべての要素が\emph{ユニーク - distinct}であることです。
したがって、関数\texttt{count}は常に
0(要素ない)か1(要素ある)を返し、
関数\texttt{insert}は要素がすでに集合にある場合、それを追加することはありません。
\begin{lstlisting}
set<int> s;
s.insert(5);
s.insert(5);
s.insert(5);
cout << s.count(5) << "\n"; // 1
\end{lstlisting}

C++には、\texttt{multiset}と\texttt{unordered\_multiset}という構造体もあり、
これらは\texttt{set} と \texttt{unordered\_set}と同様の機能しますが、
同じ値の複数要素を格納することができます。
たとえば、次のコードでは、5という数字の3つのインスタンスがすべてmultisetに追加されています。

\begin{lstlisting}
multiset<int> s;
s.insert(5);
s.insert(5);
s.insert(5);
cout << s.count(5) << "\n"; // 3
\end{lstlisting}

\texttt{erase}は複数の要素があったとしても全てを削除します。
\begin{lstlisting}
s.erase(5);
cout << s.count(5) << "\n"; // 0
\end{lstlisting}

1つの要素だけを削除したいのなら次のようにイテレータでアクセスします。
\begin{lstlisting}
s.erase(s.find(5));
cout << s.count(5) << "\n"; // 2
\end{lstlisting}

\section{map - Map structures}

\index{map}

\key{map}はキーと値のペアで構成される一般的な配列のことである。
通常の配列のキーは常に連続した整数$0,1,\ldots,n-1$(nは配列のサイズ)ですが、
マップのキーは整数でない任意のデータ型とでき、連続した値である必要はありません。

C++標準ライブラリには、
集合の実装に対応する2つのマップの実装がある。
一つは\texttt{map}で平衡二分木に基づいており、要素へのアクセスに$O(\log n)$時間かかります。
もう一つは\texttt{unordered\_map}でハッシュを使用しており、平均で$O(1)$で操作可能です。

次のコードは,キーが文字列,値が整数であるマップを作成します.
\begin{lstlisting}
map<string,int> m;
m["monkey"] = 4;
m["banana"] = 3;
m["harpsichord"] = 9;
cout << m["banana"] << "\n"; // 3
\end{lstlisting}

キーの値が要求されマップにその値が含まれていない場合、
キーは自動的にデフォルト値でマップに追加されます(訳注:参照した時点で作成されることに注意します)。
例えば、次のコードでは、値0を持つキー''aybabtu''がマップに追加されます。

\begin{lstlisting}
map<string,int> m;
cout << m["aybabtu"] << "\n"; // 0
\end{lstlisting}

\texttt{count}はキーの存在を確認します。
\begin{lstlisting}
if (m.count("aybabtu")) {
    // key exists
}
\end{lstlisting}

次のコードはキーと値を列挙します。
\begin{lstlisting}
for (auto x : m) {
    cout << x.first << " " << x.second << "\n";
}
\end{lstlisting}

\section{イテレータと範囲 - Iterators and ranges}

\index{イテレータ - iterator}

C++標準ライブラリの多くの関数は、\key{イテレータ - iterator}を使用して操作します。
イテレータはデータ構造中の要素を指す変数です(訳註:Cを知っている方はポインタと考えると良いでしょう)。

よく使われるイテレータを返す関数
\texttt{begin}と\texttt{end}
は、データ構造内の全要素を含む範囲を示すために使われます。
イテレータ\texttt{begin}はデータ構造の最初の要素を指し、\texttt{end}は最後の要素の後を指します。
\begin{center}
\begin{tabular}{llllllllll}
\{ & 3, & 4, & 6, & 8, & 12, & 13, & 14, & 17 & \} \\
& $\uparrow$ & & & & & & & & $\uparrow$ \\
& \multicolumn{3}{l}{\texttt{s.begin()}} & & & & & & \texttt{s.end()} \\
\end{tabular}
\end{center}

\texttt{s.begin()}はデータ構造内の要素を指し、
\texttt{s.end()}はデータ構造の外を指している、という非対称性に注意してください。
つまり、イテレータで定義される範囲は\emph{片方が開区間 - half-open}となっています。

\subsubsection{範囲の利用 - Working with ranges}

イテレータは、(訳註:反復した処理に用いられることが多く)
C++の標準ライブラリ関数の中でデータ構造内の要素の範囲を与えられて使用されます。
通常、データ構造内のすべての要素を処理したいので、
イテレータ\texttt{begin} と \texttt{end}が操作の関数に与えられます。

次のコードは、
\texttt{sort}関数でvectorをソートし、
\texttt{reverse}関数で要素の順序を反転させ、
\texttt{random\_shuffle}関数で要素の順序をシャッフルします。

\index{sort@\texttt{sort}}
\index{reverse@\texttt{reverse}}
\index{random\_shuffle@\texttt{random\_shuffle}}

\begin{lstlisting}
sort(v.begin(), v.end());
reverse(v.begin(), v.end());
random_shuffle(v.begin(), v.end());
\end{lstlisting}

これらは通常の配列でも使え、この場合は配列のポインタを与えます。

\newpage
\begin{lstlisting}
sort(a, a+n);
reverse(a, a+n);
random_shuffle(a, a+n);
\end{lstlisting}

\subsubsection{イテレータの設定 - Set iterators}

イテレータは、集合の要素にアクセスするためにもよく使われます。
次のコードは、集合の最小の要素を指すイテレータ\texttt{it}を作成します。
\begin{lstlisting}
set<int>::iterator it = s.begin();
\end{lstlisting}

次のように書くこともできます。
\begin{lstlisting}
auto it = s.begin();
\end{lstlisting}

イテレータが指す要素には\texttt{*}記号でアクセスすることができます。
たとえば、次のコードは、セットの最初の要素を表示します。
\begin{lstlisting}
auto it = s.begin();
cout << *it << "\n";
\end{lstlisting}

イテレータの移動は、演算子 \texttt{++}(次)
および \texttt{--}(前)を用いて行うことができ、
これはイテレータがセットの次の要素または前の要素に移動することを意味します。

これらを使って、次のように全ての配列にアクセスすることもできます。
\begin{lstlisting}
for (auto it = s.begin(); it != s.end(); it++) {
    cout << *it << "\n";
}
\end{lstlisting}

次のコードは配列の最後の要素を表示します。
\begin{lstlisting}
auto it = s.end(); it--;
cout << *it << "\n";
\end{lstlisting}

関数 $\texttt{find}(x)$ は,値が $x$ である要素を指すイテレータを返す関数ですが
集合に $x$ が含まれない場合は,イテレータは\texttt{end}を示します。
\begin{lstlisting}
auto it = s.find(x);
if (it == s.end()) {
    // x is not found
}
\end{lstlisting}

関数 $\texttt{lower\_bound}(x)$ は,集合の中で値が $x$ 以上である最小の要素へのイテレータを返し,
関数 $\texttt{upper\_bound}(x)$ は,集合の中で値が $x$ よりも大きい最小の要素へのイテレータを返します。
これらの関数は、該当する要素がないならば、\texttt{end}を返します。
また、要素の順序を保持しない\texttt{unordered\_set}構造体ではサポートされません。

\begin{samepage}
例えば、次のコードは、$x$に最も近い要素を見つけます。
\begin{lstlisting}
auto it = s.lower_bound(x);
if (it == s.begin()) {
    cout << *it << "\n";
} else if (it == s.end()) {
    it--;
    cout << *it << "\n";
} else {
    int a = *it; it--;
    int b = *it;
    if (x-b < a-x) cout << b << "\n";
    else cout << a << "\n";
}
\end{lstlisting}

このコードでは、集合が空でないことを仮定し、
イテレータ\texttt{it}を使用してすべての可能なケースを通過します。
イテレータは値が少なくとも$x$である最小の要素を指します。
もし\texttt{it}が\texttt{begin} に等しければ、対応する要素が $x$ に最も近いです。
もし\texttt{it}が\texttt{end} に等しければ、セットの中で最大の要素が $x$ に最も近いです。
もし前のケースがどれも成立しなければ、$x$に最も近い要素は前後の要素のどちらかになります。
\end{samepage}

\section{その他の構造 - Other structures}

\subsubsection{ビットセット - Bitset}

\index{ビットセット - bitset}

\key{ビットセット - bitset} とは、各値が0または1である配列です。
例えば、次のコードは10個の要素を含むビットセットを作成します。
\begin{lstlisting}
bitset<10> s;
s[1] = 1;
s[3] = 1;
s[4] = 1;
s[7] = 1;
cout << s[4] << "\n"; // 1
cout << s[5] << "\n"; // 0
\end{lstlisting}

ビットセットを使用する利点は、
ビットセットの各要素は1ビットしか要さないので少ないメモリしか必要としないことです。
\texttt{int}配列に$n$ビットを格納する場合、
$32n$ビットのメモリを使用しますが、同じ長さのビットセットは$n$ビットのメモリしか必要としません。
また、ビットセットの値はビット演算子を使って効率的に操作できるため、
ビットセットを使ったアルゴリズムの最適化が可能です。

次のコードはビットセットを作成する別の方法を示しています。
\begin{lstlisting}
bitset<10> s(string("0010011010")); // from right to left
cout << s[4] << "\n"; // 1
cout << s[5] << "\n"; // 0
\end{lstlisting}

関数 \texttt{count} は,ビットセット内の 1 の個数を返します。
\begin{lstlisting}
bitset<10> s(string("0010011010"));
cout << s.count() << "\n"; // 4
\end{lstlisting}

また、次のようにビット演算を行うことが出来ます。
\begin{lstlisting}
bitset<10> a(string("0010110110"));
bitset<10> b(string("1011011000"));
cout << (a&b) << "\n"; // 0010010000
cout << (a|b) << "\n"; // 1011111110
cout << (a^b) << "\n"; // 1001101110
\end{lstlisting}

\subsubsection{デック - Deque}

\index{デック - deque}

\key{デック - deque} は配列の両端で効率的にサイズを変更できる動的な配列です。
vectorと同様にdequeは関数 \texttt{push\_back} and \texttt{pop\_back} を提供しますが
vectorでは利用できない関数 \texttt{push\_front} and \texttt{pop\_front} も提供します。
\begin{lstlisting}
deque<int> d;
d.push_back(5); // [5]
d.push_back(2); // [5,2]
d.push_front(3); // [3,5,2]
d.pop_back(); // [3,5]
d.pop_front(); // [5]
\end{lstlisting}

dequeの内部実装はvectorよりも複雑なので遅いですが、
要素の追加と削除は両端とも平均して$O(1)$です。

\subsubsection{スタック - Stack}

\index{スタック - stack}

\key{スタック - stack}は$O(1)$時間の操作を提供するデータ構造で、
要素をトップに追加すること、トップから要素を削除することの2つの操作だけを提供します。
つまり、スタックの先頭の要素にのみアクセスすることができる構造体です。

次のコードは、スタックを使用する方法を示しています。
\begin{lstlisting}
stack<int> s;
s.push(3);
s.push(2);
s.push(5);
cout << s.top(); // 5
s.pop();
cout << s.top(); // 2
\end{lstlisting}
\subsubsection{Queue}

\index{キュー - queue}

\key{キュー - queue} も末尾に要素を追加する操作と、
待ち行列の先頭の要素を削除する操作の2つの$O(1)$の操作だけが用意されています。
つまり、待ち行列の最初と最後の要素にのみアクセスすることができます。

次のコードは、キューを使用する方法を示しています。

\begin{lstlisting}
queue<int> q;
q.push(3);
q.push(2);
q.push(5);
cout << q.front(); // 3
q.pop();
cout << q.front(); // 2
\end{lstlisting}

\subsubsection{優先度付きキュー - Priority queue}

\index{優先度付きキュー - priority queue}
\index{ヒープ - heap}

\key{優先度付きキュー - priority queue}は要素の集合を保持するデータ構造です。
サポートされている操作は、挿入と、待ち行列の種類に応じて、最小または最大の要素の検索と削除です。
挿入と削除には $O(\log n)$ の時間がかかり、取り出しには $O(1)$ の時間がかかります。。
setでも、優先度付きキューでできるすべての操作が可能ですが、
優先度付きキューは定数係数がより小さく操作できます。
優先度付きキューはヒープ構造を用いて実装されますが、
順序付きセットで用いられる平衡二分木よりもはるかに単純なものです。

\begin{samepage}

C++の優先度付きキューは、デフォルトは要素の降順でソートされ、キュー内の最大要素を見つけ、削除することが可能です。
\begin{lstlisting}
priority_queue<int> q;
q.push(3);
q.push(5);
q.push(7);
q.push(2);
cout << q.top() << "\n"; // 7
q.pop();
cout << q.top() << "\n"; // 5
q.pop();
q.push(6);
cout << q.top() << "\n"; // 6
q.pop();
\end{lstlisting}
\end{samepage}

昇順(つまり小さい順)の優先度付きキューを作りたいならば次のようにします。
\begin{lstlisting}
priority_queue<int,vector<int>,greater<int>> q;
\end{lstlisting}

\subsubsection{Policy-based data structures}

\texttt{g++} は、
C++標準ライブラリに含まれないデータ構造もいくつかサポートしています。
このような構造は、\emph{policy-based}データ構造と呼ばれています。
これらの構造体を使うには、次の行をコードに追加する必要があります。

\begin{lstlisting}
#include <ext/pb_ds/assoc_container.hpp>
using namespace __gnu_pbds; 
\end{lstlisting}

これで\texttt{set}のように配列のようにインデックスを付けられる
データ構造\texttt{indexed\_set}を定義することができるようになりました。
\texttt{int}を保持する構造は以下のように作ります。
\begin{lstlisting}
typedef tree<int,null_type,less<int>,rb_tree_tag,
             tree_order_statistics_node_update> indexed_set; 
\end{lstlisting}

次のように作ります。

\begin{lstlisting}
indexed_set s;
s.insert(2);
s.insert(3);
s.insert(7);
s.insert(9);
\end{lstlisting}

この拡張setはソートされた配列の要素にあるインデックスにアクセスできます。
関数 $\texttt{find\_by\_order}$ は,指定された位置の要素へのイテレータを返します。
\begin{lstlisting}
auto x = s.find_by_order(2);
cout << *x << "\n"; // 7
\end{lstlisting}

$\texttt{order\_of\_key}$は与えた要素のindexを返します。
(TODO: これ本当か?2ではない気もする)
\begin{lstlisting}
cout << s.order_of_key(7) << "\n"; // 2
\end{lstlisting}

その要素が集合に現れない場合、その要素が集合の中で持つであろう位置を求めます。
\begin{lstlisting}
cout << s.order_of_key(6) << "\n"; // 2
cout << s.order_of_key(8) << "\n"; // 3
\end{lstlisting}

これらは対数時間で動作します。

\section{ソートとの比較 - Comparison to sorting}

データ構造とソートのどちらかを使って解ける問題は多いです。
これらのアプローチの実際の効率に顕著な違いがでることもあり、
それは時間的な複雑さに起因することがあります。

$n$個の要素を含む2つのリスト$A$、$B$が与えられたときの問題を考えてみましょう。ここで両方に所属する要素数をカウント死体とします。

例えば、以下の例を考えます。
\[A = [5,2,8,9] \hspace{10px} \textrm{and} \hspace{10px} B = [3,2,9,5],\]
この場合、2,5,9が両方に含まれるので答えは3です。

愚直な比較を行うと$O(n^2)$ですが、いくつかのアプローチを考えてみましょう。

\subsubsection{方法1: Algorithm 1}

$A$のsetをつくり、$B$の各要素について$A$にも属するかどうかをチェックします。
時間計算量は$O(n \log n)$です。

\subsubsection{方法2: Algorithm 2}

順序付きセットである必要がないので
\texttt{set}の代わりに\texttt{unordered\_set}を使うこともできます。
基礎となるデータ構造を変更するだけなので非常に簡単な方法です。
時間計算量は$O(n)$になりました。

\subsubsection{方法3: Algorithm 3}

ソートを使うことができます。
まず$A$ と$B$をソートします。
この後、両方のリストを同時に反復処理し、共通の要素を見つけます。
ソートの時間計算量は$O(n \log n)$で
残りのアルゴリズムも$O(n)$ 時間で動作するので、全体の時間計算量は$O(n \log n)$となります。

\subsubsection{効率性の比較 - Efficiency comparison}

次の表は$n$が変化し,リストの要素が $1 \ldots 10^9$ の間のランダムな整数である場合に,
上記のアルゴリズムがどの程度効率的に動作するかを示します。
\begin{center}
\begin{tabular}{rrrr}
$n$ & 方法 1 & 方法 2 & 方法 3 \\
\hline
$10^6$ & $1.5$ s & $0.3$ s & $0.2$ s \\
$2 \cdot 10^6$ & $3.7$ s & $0.8$ s & $0.3$ s \\
$3 \cdot 10^6$ & $5.7$ s & $1.3$ s & $0.5$ s \\
$4 \cdot 10^6$ & $7.7$ s & $1.7$ s & $0.7$ s \\
$5 \cdot 10^6$ & $10.0$ s & $2.3$ s & $0.9$ s \\
\end{tabular}
\end{center}

アルゴリズム1と2は、使用する集合構造が異なることを除けば同じコードです。
この問題では、この選択が実行時間に重要な影響を及ぼします。
アルゴリズム2はアルゴリズム1より4-5倍速く動作します。

しかし、最も効率的な方法は、ソートを使用する方法3であることに注意してください。
方法2に比べ、半分の時間しか使いません。
興味深いことに、方法1と方法3の時間計算量はともに$O(n \log n)$ですが、
それにもかかわらず方法3は10倍も速いのです。

これは、ソートが方法3の冒頭で一度だけ行われるのに対して、
残りの方法は線形時間で動作するためです。

一方、方法1はアルゴリズム全体の間、複雑なバランス二分木を維持するために
毎回$O(\log n)$の動作を行うためです。